\documentclass[11pt, a4paper]{article}

\usepackage{amsmath}
\usepackage{amssymb}
\usepackage{amsthm}
\usepackage{IEEEtrantools}
\usepackage{CJKutf8}
\usepackage{geometry}
\geometry{margin = 1cm}
\usepackage{enumitem}

\title{Linear Algebra Homework 3}
\author{107305008}
\date{}

\begin{document}
\maketitle
\begin{center}
\begin{CJK}{UTF8}{bsmi}
財管二 孫偉翔
\end{CJK}
\end{center}
\pagenumbering{gobble}
\begin{description}
\item {\Large{\textbf{Exercise 6.7}}}
\begin{description}
	\item Problem 13
	\begin{description}
		\item{(a)}
		find $[v]_T $ 
		\begin{IEEEeqnarray*}{rCl}
		\begin{bmatrix}
		1 & 2\\
		1 & 3
		\end{bmatrix}
		\begin{bmatrix}
		c_1 \\
		c_2
		\end{bmatrix}
		&=&
		\begin{bmatrix}
		1\\
		5
		\end{bmatrix}\\
		\rightarrow
		\begin{bmatrix}
		1 & 0\\
		0 & 1
		\end{bmatrix}
		\begin{bmatrix}
		c_1 \\
		c_2
		\end{bmatrix}
		&=&
		\begin{bmatrix}
		-7\\
		4
		\end{bmatrix},\qquad %
		[v]_T = 
		\begin{bmatrix}
		c_1 \\
		c_2
		\end{bmatrix} = 
		\begin{bmatrix}
		-7\\
		4
		\end{bmatrix}\sharp
		\end{IEEEeqnarray*}
		%%%%%%%%%%%%%%%%%%%%%%%%%%%%%%%%%%
		find $[w]_T $ 
		\begin{IEEEeqnarray*}{rCl}
		\begin{bmatrix}
		1 & 2\\
		1 & 3
		\end{bmatrix}
		\begin{bmatrix}
		c_1 \\
		c_2
		\end{bmatrix}
		&=&
		\begin{bmatrix}
		5\\
		4
		\end{bmatrix}\\
		\rightarrow
		\begin{bmatrix}
		1 & 0\\
		0 & 1
		\end{bmatrix}
		\begin{bmatrix}
		c_1 \\
		c_2
		\end{bmatrix}
		&=&
		\begin{bmatrix}
		7\\
		-1
		\end{bmatrix},\qquad %
		[w]_T = 
		\begin{bmatrix}
		c_1 \\
		c_2
		\end{bmatrix} = 
		\begin{bmatrix}
		7\\
		-1
		\end{bmatrix}\sharp
		\end{IEEEeqnarray*}
		\item{(b)} find $P_{S\leftarrow T}$
		\begin{IEEEeqnarray*}{rCl}
			P_{S\leftarrow T} &=& [[T_1]_S\ [T_2]_S]\\
		\end{IEEEeqnarray*}
		\begin{IEEEeqnarray*}{rCl}
		\begin{bmatrix}
		1 & 0\\
		2 & 1
		\end{bmatrix}
		\begin{bmatrix}
		c_1 & c_3 \\
		c_2 & c_4
		\end{bmatrix}
		&=&
		\begin{bmatrix}
		1 & 2\\
		1 & 3
		\end{bmatrix}\\
		%%%%%%%%%%%%%%%%%%
		\rightarrow
		\begin{bmatrix}
		1 & 0\\
		0 & 1
		\end{bmatrix}
		\begin{bmatrix}
		c_1 & c_3 \\
		c_2 & c_4
		\end{bmatrix}
		&=&
		\begin{bmatrix}
		1 & 2\\
		-1 & -1
		\end{bmatrix}\\
		P_{S\leftarrow T} &=&
		\begin{bmatrix}
		1 & 2\\
		-1 & -1
		\end{bmatrix}\sharp
		\end{IEEEeqnarray*}
		\item{(c)} find $[v]_S,\ [w]_S$
			\begin{IEEEeqnarray*}{rCl}
				[v]_S &=& P_{S\leftarrow T}[v]_T\\
				&=& %
				\begin{bmatrix}
				1&2\\
				-1&-1
				\end{bmatrix}
				\begin{bmatrix}
				-7\\
				4
				\end{bmatrix}\\
				&=& 
				\begin{bmatrix}
				1\\
				3
				\end{bmatrix} \sharp
			\end{IEEEeqnarray*}
			\begin{IEEEeqnarray*}{rCl}
				[w]_S &=& P_{S\leftarrow T}[w]_T\\
				&=& %
				\begin{bmatrix}
				1&2\\
				-1&-1
				\end{bmatrix}
				\begin{bmatrix}
				7\\
				-1
				\end{bmatrix}\\
				&=& 
				\begin{bmatrix}
				5\\
				-6
				\end{bmatrix} \sharp
			\end{IEEEeqnarray*}
		\item{(d)} find $[w]_S$ and $[v]_S$
			\begin{equation*}
				\begin{bmatrix}
				S_1 & S_2 
				\end{bmatrix}
				\begin{bmatrix}
				[v]_S [w]_S
				\end{bmatrix}
				= %
				\begin{bmatrix}
				v & w 
				\end{bmatrix}
			\end{equation*}
			\begin{IEEEeqnarray*}{rCl}
				\begin{bmatrix}
				1 & 0\\
				2 & 1
				\end{bmatrix}
				\begin{bmatrix}
				c_1 & c_3\\
				c_2 & c_4
				\end{bmatrix}
				&=& %
				\begin{bmatrix}
				1 & 5 \\
				5 & 4 
				\end{bmatrix}\\
				\begin{bmatrix}
				1 & 0\\
				0 & 1
				\end{bmatrix}
				\begin{bmatrix}
				c_1 & c_3\\
				c_2 & c_4
				\end{bmatrix}
				&=&
				\begin{bmatrix}
				1 & 5\\
				3 & -6
				\end{bmatrix}\qquad , [v]_S = %
				 \begin{bmatrix}
				1 \\
				3 
				\end{bmatrix} \quad [w]_S =%
				 \begin{bmatrix}
				5 \\
				-6 
				\end{bmatrix}\sharp
			\end{IEEEeqnarray*}
		\item{(e)} find $Q_{T\leftarrow S}$
		$$Q_{T\leftarrow S} = [[S_1]_T [S_2]_T]$$
		\begin{IEEEeqnarray*}{rCl}
			\begin{bmatrix}
			1 & 2\\
			1 & 3
			\end{bmatrix}
			[[S_1]_T[S_2]_T]%
			&=& 
			\begin{bmatrix}
			1 & 0\\
			2 & 1
			\end{bmatrix}\\
			%%%%%%%%%%%%%%%%%%%
			\rightarrow
			\begin{bmatrix}
			1 & 0\\
			0 & 1
			\end{bmatrix}
			[[S_1]_T[S_2]_T]%
			&=& 
			\begin{bmatrix}
			-1 & -2\\
			1 & 1
			\end{bmatrix} \qquad Q_{T\leftarrow S} =
			\begin{bmatrix}
			-1 & -2\\
			1 & 1
			\end{bmatrix}\sharp
		\end{IEEEeqnarray*}
		\item{(f)} find $[v]_T$ and $[w]_T$
		\begin{IEEEeqnarray*}{rCl}
		[[v]_T[w]_T] &=& Q_{T\leftarrow S}[[v]_S[w]_S]\\
		&=&%
		\begin{bmatrix}
			-1 & -2\\
			1 & 1
		\end{bmatrix}
		\begin{bmatrix}
		1 & 5\\
		3 & -6
		\end{bmatrix}\\
		&=&			
		\begin{bmatrix}
		-7 & 7\\
		4 & -1
		\end{bmatrix}\qquad [v]_T=
		\begin{bmatrix}
		-7\\
		4
		\end{bmatrix} [w]_T=
		\begin{bmatrix}
		7\\
		-1
		\end{bmatrix}\sharp
		\end{IEEEeqnarray*}
		\text{which corresponds to the result in (a)}
	\end{description}
	\item Problem 15\\
	given bases
	$S = \{t^2+1,t-2,t+3\}, T=\{2t^2+t,t^2+3,t\}$ for $P_2$\\
	\phantom{given bases }$v=8t^2-4t+6$, $w = 7t^2 -t +9$
	\begin{description}
	\item(a) find $[v]_T, [w]_T$
	\begin{IEEEeqnarray*}{rCl}
		\begin{bmatrix}
		T_1 & T_2 & T_3
		\end{bmatrix}
		\begin{bmatrix}
		[v]_T & [w]_T
		\end{bmatrix} &=&
		\begin{bmatrix}
		v & w
\end{bmatrix}				
	\end{IEEEeqnarray*}
	\begin{IEEEeqnarray*}{rCl}
	\begin{bmatrix}
	2&1&0\\1&0&1\\0&3&0
	\end{bmatrix}
	\begin{bmatrix}
	[v]_T & [w]_T
	\end{bmatrix}&=& 
	\begin{bmatrix}
	8&7\\-4&-1\\6&9
	\end{bmatrix}\\
	\rightarrow
	\begin{bmatrix}
	1&0&0\\0&1&0\\0&0&1
	\end{bmatrix}
	\begin{bmatrix}
	[v]_T & [w]_T
	\end{bmatrix}&=&
	\begin{bmatrix}
	3&2\\2&3\\-7&-3
	\end{bmatrix}
	\end{IEEEeqnarray*}
	$[v]_T = 
	\begin{bmatrix}
	3\\2\\-7
	\end{bmatrix},[w]_T =
	\begin{bmatrix}
	2\\3\\-3
	\end{bmatrix}\sharp$
	\item(b)
	\begin{IEEEeqnarray*}{rCl}
	P_{S\leftarrow T} & = &\begin{bmatrix}
	[W_1]_S & [W_2]_S &[W_3]_S
	\end{bmatrix}
	\end{IEEEeqnarray*}
	\begin{IEEEeqnarray*}{rCl}
	M_SP_{S\leftarrow T}&=&M_T\\
	\rightarrow\begin{bmatrix}
	1&0&0\\0&1&1\\1&-2&3
	\end{bmatrix}P_{S\leftarrow T}&=&\begin{bmatrix}
	2&1&0\\1&0&1\\0&3&0
	\end{bmatrix}\\
	\rightarrow 
	\begin{bmatrix}
	1&0&0\\0&1&0\\0&0&1
	\end{bmatrix}
	P_{S\leftarrow T}&=&
	\begin{bmatrix}
	2&1&0\\1&-0.4&0.6\\0&0.4&0.4
	\end{bmatrix}\\
	\rightarrow P_{S\leftarrow T}&=&
	\begin{bmatrix}
	2&1&0\\1&-0.4&0.6\\0&0.4&0.4
	\end{bmatrix}\sharp
	\end{IEEEeqnarray*}
	%%%%%%%%%%%%%%%%%%%%%%%%%
	\item(c)
	\begin{IEEEeqnarray*}{rCl}
	\begin{bmatrix}
	[v]_S&[w]_S
	\end{bmatrix}&=&P_{S\leftarrow T}
	\begin{bmatrix}
	[v]_T&[w]_T
	\end{bmatrix}\\
	\rightarrow
	\begin{bmatrix}
	[v]_S&[w]_S
	\end{bmatrix}&=&\begin{bmatrix}
	2&1&0\\1&-0.4&0.6\\0&0.4&0.4
	\end{bmatrix}
	\begin{bmatrix}
	3&2\\2&3\\-7&-3
	\end{bmatrix}\\
	&=&\begin{bmatrix}
	8&7\\-2&-1\\-2&0
	\end{bmatrix}\qquad, [v]_S=\begin{bmatrix}
	8\\-2\\-2
\end{bmatrix} [w]_S=\begin{bmatrix}
7\\-1\\0
\end{bmatrix}\sharp
	\end{IEEEeqnarray*}
	\item(d)
	\begin{IEEEeqnarray*}{rCl}
	\begin{bmatrix}
	1&0&0\\0&1&1\\1&-2&3
	\end{bmatrix}\begin{bmatrix}
	[v]_S&[w]_S
	\end{bmatrix}&=&
	\begin{bmatrix}
	8&7\\-4&-1\\6&9
	\end{bmatrix}\\
	\end{IEEEeqnarray*}
	$\rightarrow [v]_S=\begin{bmatrix}
	-8\\-2\\-2
\end{bmatrix} [w]_S=\begin{bmatrix}
7\\-1\\0
\end{bmatrix}\sharp$	
	\item(e)
	\begin{IEEEeqnarray*}{rCl}
	M_SQ_{T\leftarrow S}&=&M_T\\
	\begin{bmatrix}
	2&1&0\\1&0&1\\0&3&0
	\end{bmatrix}Q_{T\leftarrow S}&=&\begin{bmatrix}
	1&0&0\\0&1&1\\1&-2&3
	\end{bmatrix}\\
	\begin{bmatrix}
	1&0&0\\0&1&0\\0&0&1
	\end{bmatrix}Q_{T\leftarrow S}&=&\begin{bmatrix}
	1/3&1/3&-0.5\\1/3&-2/3&1\\-1/3&2/3&1.5
	\end{bmatrix}\sharp
	\end{IEEEeqnarray*}
	
	\item(f)
	\begin{IEEEeqnarray*}{rCl}
	[[v]_T[w]_T]&=&P_{S\rightarrow T}[[v]_S[w]_S]\\
	\rightarrow [[v]_T[w]_T]&=&	\begin{bmatrix}
	3&2\\2&3\\-7&-3
	\end{bmatrix}
	\end{IEEEeqnarray*}
	which corresponds to the result of (a)$\sharp$
	\end{description}
	\item Problem 17
	\begin{description}
		\item (a)
		\begin{IEEEeqnarray*}{rCl}
		\begin{bmatrix}
		1&0&0&1\\0&1&0&0\\1&0&0&0\\1&0&1&0
		\end{bmatrix}[[v]_T[w]_T]&=&\begin{bmatrix}
		1&1\\1&-2\\1&2\\1&1\\
		\end{bmatrix}\\
		\rightarrow \begin{bmatrix}
		1&0&0&0\\0&1&0&0\\0&0&1&0\\0&0&0&1
		\end{bmatrix}[[v]_T[w]_T]&=&\begin{bmatrix}
		1&2\\1&-2\\1&1\\0&-1\\
		\end{bmatrix}
		\end{IEEEeqnarray*}
		$[v]_T = \begin{bmatrix}
		1\\1\\1\\0
		\end{bmatrix}[w]_T=\begin{bmatrix}
		2\\-2\\1\\-1
		\end{bmatrix}\sharp$
		\item (b)
		\begin{IEEEeqnarray*}{rCl}
		\begin{bmatrix}
		1&0&0&0\\0&1&0&1\\0&1&2&0\\0&0&1&1\\
\end{bmatrix}P_{S\leftarrow T}&=&\begin{bmatrix}
		1&0&0&1\\0&1&0&0\\1&0&0&0\\1&0&1&0
		\end{bmatrix}\\
		\rightarrow
		\begin{bmatrix}
		1&0&0&0\\0&1&0&0\\0&0&1&0\\0&0&0&1
		\end{bmatrix}P_{S\leftarrow T}&=&\begin{bmatrix}
		1&0&0&1\\1/3&2/3&-2/3&0\\1/3&-1/3&1/3&0\\-1/3&1/3&2/3&0
		\end{bmatrix}\\
		\end{IEEEeqnarray*}
		$P_{S\leftarrow T} = \begin{bmatrix}
		1&0&0&1\\1/3&2/3&-2/3&0\\1/3&-1/3&1/3&0\\-1/3&1/3&2/3&0
		\end{bmatrix}\sharp$
		\item (c)
		\begin{IEEEeqnarray*}{rCl}
		[[v]_S[w]_S]&=&P_{S\leftarrow T}\begin{bmatrix}
		1&2\\1&-2\\1&1\\0&-1\\
		\end{bmatrix}\\
		&=&\begin{bmatrix}
		1&0&0&1\\1/3&2/3&-2/3&0\\1/3&-1/3&1/3&0\\-1/3&1/3&2/3&0
		\end{bmatrix}\begin{bmatrix}
		1&2\\1&-2\\1&1\\0&-1\\
		\end{bmatrix}\\
		&=&\begin{bmatrix}
		1&1\\1/3&-4/3\\1/3&5/3\\2/3&-2/3
		\end{bmatrix}\qquad,[v]_S=\begin{bmatrix}
		1\\1/3\\1/3\\2/3
		\end{bmatrix}[w]_S=\begin{bmatrix}
		1\\-4/3\\5/3\\-2/3
		\end{bmatrix}\sharp
		\end{IEEEeqnarray*}
		\item (d)
		\begin{IEEEeqnarray*}{rCl}
		\begin{bmatrix}
		1&0&0&0\\0&1&0&1\\0&1&2&0\\0&0&1&1\\
\end{bmatrix}[[v]_S[w]_S]&=&\begin{bmatrix}
		1&1\\1&-2\\1&2\\1&1\\
		\end{bmatrix}\\
		\rightarrow
		\begin{bmatrix}
		1&0&0&1\\0&1&0&0\\1&0&0&0\\1&0&1&0
		\end{bmatrix}[[v]_S[w]_S] &=&\begin{bmatrix}
		1&1\\1/3&-4/3\\1/3&5/3\\2/3&-2/3
		\end{bmatrix}\qquad,[v]_S=\begin{bmatrix}
		1\\1/3\\1/3\\2/3
		\end{bmatrix}[w]_S=\begin{bmatrix}
		1\\-4/3\\5/3\\-2/3
		\end{bmatrix}\sharp
		\end{IEEEeqnarray*}
		\item (e)
		\begin{IEEEeqnarray*}{rCl}
		\begin{bmatrix}
		1&0&0&1\\0&1&0&0\\1&0&0&0\\1&0&1&0
\end{bmatrix}Q_{T\leftarrow S}&=&\begin{bmatrix}
		1&0&0&0\\0&1&0&1\\0&1&2&0\\0&0&1&1
		\end{bmatrix}\\
		\rightarrow
		\begin{bmatrix}
		1&0&0&0\\0&1&0&0\\0&0&1&0\\0&0&0&1
		\end{bmatrix}Q_{T\leftarrow S}&=&\begin{bmatrix}
		0&1&2&0\\0&1&0&1\\0&0&1&1\\1&-1&-2&0
		\end{bmatrix}\\
		\rightarrow Q_{T\leftarrow S}&=&\begin{bmatrix}
		0&1&2&0\\0&1&0&1\\0&0&1&1\\1&-1&-2&0\end{bmatrix}\sharp
		\end{IEEEeqnarray*}
		\item (f)
		\begin{IEEEeqnarray*}{rCl}
		[[v]_T[w]_T]&=&\begin{bmatrix}
		0&1&2&0\\0&1&0&1\\0&0&1&1\\1&-1&-2&0\end{bmatrix}\begin{bmatrix}
		1&1\\1/3&-4/3\\1/3&5/3\\2/3&-2/3
		\end{bmatrix}\\
		&=&\begin{bmatrix}
		1&2\\1&-2\\1&1\\0&-1\\
		\end{bmatrix}
		\end{IEEEeqnarray*}
		which corresponds to the result in (a)$\sharp$
	\end{description}
	\item Problem 23
	\begin{IEEEeqnarray*}{rCl}
	P_{S\leftarrow T}& = &
	\begin{bmatrix}
	1 & 1 & 2\\
	2 & 1 & 1\\
	-1 & -1& 1
	\end{bmatrix}\\
	&=&[[w_1]_S[w_2]_S[w_3]_S]
	\end{IEEEeqnarray*}
	thus it follows that
	\begin{IEEEeqnarray*}{rCl}
	\begin{bmatrix}
	v_1 & v_2 & v_3
	\end{bmatrix}
	[[w_1]_S[w_2]_S[w_3]_S] &=&
	\begin{bmatrix}
	w_1 & w_2 & w_3
	\end{bmatrix}\\
	\rightarrow
	\begin{bmatrix}
	1&1&0\\
	0&1&0\\
	1&0&1
	\end{bmatrix}
	\begin{bmatrix}
	1 & 1 & 2\\
	2 & 1 & 1\\
	-1 & -1& 1
	\end{bmatrix} &=& 
	\begin{bmatrix}
	w_1 & w_2 & w_3
	\end{bmatrix}\\
	\rightarrow
	\begin{bmatrix}
	3 & 2 & 3\\
	2 & 1 & 1\\
	0 & 0 & 3
	\end{bmatrix} &=& 
	\begin{bmatrix}
	w_1 & w_2 & w_3
	\end{bmatrix}
	\end{IEEEeqnarray*}
	the set
	\begin{equation}
	T = \left\{ 
	\begin{bmatrix}
	3\\
	2\\
	0
	\end{bmatrix},
	\begin{bmatrix}
	2\\
	1\\
	0
	\end{bmatrix},
	\begin{bmatrix}
	3\\
	1\\
	3
	\end{bmatrix}\right\} \sharp
	\end{equation}
	\item Problem 26
	find $S$ $$P_{S\leftarrow T} = \begin{bmatrix}
	1&2\\2&3
	\end{bmatrix}=[[w_1]_S[w_2]_S]
	$$
	\begin{IEEEeqnarray*}{rCl}
	\begin{bmatrix}
	v_1 & v_2
	\end{bmatrix}
	[[w_1]_S[w_2]_S] &=& \begin{bmatrix}
	w_1 & w_2
	\end{bmatrix}\\
	\rightarrow
	\begin{bmatrix}
	v_1 & v_2
	\end{bmatrix}\begin{bmatrix}
	1&2\\2&3
	\end{bmatrix} &=&
	\begin{bmatrix}
	1&1\\-1&1
	\end{bmatrix}\\
	%%%%%%%%%%%%%%%%%%%%%
	\rightarrow
	\begin{bmatrix}
	v_1 & v_2
	\end{bmatrix}
	&=&
	\begin{bmatrix}
	1&1\\-1&1
	\end{bmatrix}
	\begin{bmatrix}
	1&2\\2&3
	\end{bmatrix}^{-1}\\
	&=&\begin{bmatrix}
	1&1\\-1&1
	\end{bmatrix}\begin{bmatrix}
	-3&2\\2&-1
	\end{bmatrix}\\
	&=&\begin{bmatrix}
	-1&1\\5&-3
	\end{bmatrix}
	\end{IEEEeqnarray*}
	$$S=\left\{-t+5,t-3\right\}\sharp$$
\end{description}
\end{description}
\begin{description}
\item {\Large{\textbf{Theoretical Exercise 6.7}}}
\begin{description}
	\item T.1.
	\begin{proof}
	suppose $v=w$, since $S$ is a basis, there can only exist a unique set of coefficient $\{c_1, c_2\ldots c_n\}$, such that $c_1v_1 + c_2v_2 + \ldots +c_nv_n = v$, and also, there can only exist a unique set of coefficient $\{b_1, b_2\ldots b_n\}$, such that $b_1v_1 + b_2v_2 + \ldots +b_nv_n = w$. And since $v=w$, it follows that, $c_1=b_1, c_2=b_2 \ldots, c_n=b_n$, adnd thus, $[v]_S = [w]_S$. On the contrary, suppose $[v]_S = [w]_S$, and thus, the coefficient that form $v$ and $w$ are the same, meaning $c_1v_1 + c_2v_2 + \ldots +c_nv_n = v = w$, thus, $w=v$.\\
	As a result, $v=w$ if and only if $[v]_S = [w]_S$
	\end{proof}
	
	\item T.3.
	\begin{proof}
	suppose $\{w_1,w_2\ldots,w_k\}$ is a linearly independent set of vectors in $V$. It follows that $c_1w_1 + c_2w_2 \ldots + c_kw_k = \vec{0}$, where only $c_1=c_2=\ldots=c_k=0$ satisfies the equation. Changing both sides to base $S$ coordinate vector. It follows that,
	\begin{IEEEeqnarray*}{rCl}
	[c_1w_1 + c_2w_2 \ldots + c_kw_k]_S &=& [\vec{0}]_S\\
	\rightarrow [c_1w_1]_S + [c_2w_2]_S \ldots + [c_kw_k]_S &=&0_n\\
	\rightarrow c_1[w_1]_S + c_2[w_2]_S \ldots + c_k[w_k]_S &=&0_n
	\end{IEEEeqnarray*}
	 \text{where only } $c_1=c_2=\ldots=c_k=0$ \text{ satifies the equation}, thus, ${[w_1]_S,[w_2]_S,\ldots[w_k]_S}$ a is linearly independent set of vectors in $\mathbb{R}^n$
	
	\end{proof}
	\item T.4.
	\begin{proof}
	Since $S$ is a basis, it follows that $$[v_1]_S=e_1,[v_2]_S=e_2, \ldots,[v_n]_S=e_n$$ Thus, $$\{[v_1]_S,[v_2]_S\ldots[v_n]_S\}=\{e_1,e_2\ldots,e_n\}$$. Since every vector in $\mathbb{R}^n$ can be spanned by $\{e_1,e_2\ldots,e_n\}$, and the vectors in $\{e_1,e_2\ldots,e_n\}$ are linearly independent. Thus $\{[v_1],[v_2]\ldots[v_n]_S\}$ is a basis $\mathbb{R}^n$.
	\end{proof}
	\item T.7.
	\begin{description}
		\item{(a)}\begin{proof}
		since 
		\begin{IEEEeqnarray*}{rCl}
		M_S &= & \begin{bmatrix}
		v_1 & v_2 &\ldots &v_n
		\end{bmatrix}	\\
		M_T &= & \begin{bmatrix}
		w_1 & w_2 &\ldots &w_n
		\end{bmatrix}		 	 
		\end{IEEEeqnarray*}
		and it is known that, 
		\begin{IEEEeqnarray*}{rCl}
		P_{S\leftarrow T} &=& [[v_1]_S[v_2]_S\ldots[v_3]_S]\qquad\text{and that }\\
		M_S[w_1]_S &=& w_1
		\end{IEEEeqnarray*}
		thus, 
		\begin{IEEEeqnarray*}{rCl}
		M_S[[v_1]_S[v_2]_S\ldots[v_3]_S]&=&M_T\\
		\rightarrow M_SP_{S\leftarrow T} &=& M_T
		\end{IEEEeqnarray*}
		and since $M_T$ and $M_S$ are non-singular, thus it follows that, $P_{S\leftarrow T} = M_S^{-1}M_T\sharp$
		\end{proof}
		\item{(b)}\begin{proof}
		consider homogeneous system $P_{S\leftarrow T}\vec{x}=\vec{0}$ it means that 
		\begin{IEEEeqnarray*}{rCl}
		M_S^{-1}M_T\vec{x}&=&\vec{0}\qquad\text{by multiplying $M_S$ on both sides}\\
		\rightarrow M_T\vec{x} &=& \vec{0}
		\end{IEEEeqnarray*}
		and since $M_T$ is non-singular, it follows that the homogeneous system only contains trivial solutions. Thus, $P_{S\leftarrow T}$ is non-singular
		\end{proof}
		\item{(c)}\begin{proof}
		\begin{IEEEeqnarray*}{rCl}
		M_S &=& \begin{bmatrix}
		2&1&1\\0&2&1\\1&0&1
		\end{bmatrix}\\
		\rightarrow M_S^{-1} &=&
		\frac{1}{3}\begin{bmatrix}
		2&-1&-1&\\1&1&-2\\-2&1&4
		\end{bmatrix}\\
		M_T &=&\begin{bmatrix}
		6&4&5\\3&-1&5\\3&3&2
		\end{bmatrix}\\
		P_{S\leftarrow T}&=& M_S^{-1}M_T\\
		&=&\frac{1}{3}\begin{bmatrix}
		2&-1&-1&\\1&1&-2\\-2&1&4
		\end{bmatrix}
		\begin{bmatrix}
		6&4&5\\3&-1&5\\3&3&2
		\end{bmatrix}\\
		&=&
		\begin{bmatrix}
		2&2&1\\1&-1&2\\1&1&1
		\end{bmatrix}
		\end{IEEEeqnarray*}
		the result holds $\sharp$
		\end{proof}
	\end{description}
\end{description}
\end{description}

\end{document}

















